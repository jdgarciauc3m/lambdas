\section{Parameters and return values}

\begin{frame}[t,fragile]{Omitting parameters}
  \begin{itemize}
    \item If the lambda takes no parameter, the parenthesis can be omitted.
  \end{itemize}
\begin{lstlisting}
std::thread t([](){ do_something(); });

std::thread t([]{ do_something(); });
\end{lstlisting}
\end{frame}

\begin{frame}[t,fragile]{Specifying the return type}
  \begin{itemize}
    \item The \textgood{return type} of a lambda may be 
          \textmark{explictly specified}.
    \item \textmark{Postfix notation} is used to specify the \textgood{return type}.
  \end{itemize}
\pause
\begin{lstlisting}
std::sort(v.begin(), v.end(), 
  [](const record & x, const record & y) -> bool
  { 
    return x.id < y.id;
  }
);

std::sort(v.begin(), v.end(), 
  [](const record & x, const record & y) // return type deduced
  { 
    return x.id < y.id;
  }
);

\end{lstlisting}
\end{frame}

\begin{frame}[t,fragile]{Omitting the return type}
\begin{itemize}
  \item \textmark{Return type} can be \textgood{automatically deduced}
        when the body has a \textmark{return statement}.
\end{itemize}

\pause
\begin{itemize}
  \item \textenum{Rules}:
  \begin{itemize}
    \item The \textmark{return type} is the \textgood{type of the expression}.
    \item If needed, the following conversions are applied:
      \begin{itemize}
        \item \textmark{l-value} to \textmark{r-value}.
        \item \textmark{array} to \textmark{pointer}.
        \item \textmark{function} to \textmark{function pointer}.
      \end{itemize}
  \end{itemize}
\end{itemize}

\begin{lstlisting}
[&] { return x+y; }
[&] () -> decltype(x+y) { return x+y; }
\end{lstlisting}

\pause
  \begin{itemize}
    \item Otherwise, \textmark{return type} is deemed to be \cppkey{void}.
  \end{itemize}
\begin{lstlisting}
[] { do_something(); }
[] () -> void { do_something(); }
\end{lstlisting}
\end{frame}


