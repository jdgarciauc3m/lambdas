\section{What is a lambda expression?}

\begin{frame}[t,fragile]{Function objects in classic C++}
\mode<presentation>{\vspace{-1em}}
\begin{itemize}
  \item A \textmark{function object} is an object that may behave as a function.
    \begin{itemize}
      \item It has a member \cppkey{operator()}.
    \end{itemize}
\end{itemize}

\begin{block}{Defining a function object type}
\begin{lstlisting}
struct square {
  double operator()(double x) const noexcept { return x*x; }
};
\end{lstlisting}
\end{block}

\mode<presentation>{\vfill\pause}
\begin{itemize}
  \item It can be passed to another function.
\begin{block}{Using a function object}
\begin{lstlisting}
void f(std::vector<double> & v, std::vector<double> & w) {
  square sq;
  std::transform(std::begin(v), std::end(v), std::begin(v), sq);
  std::transform(std::begin(w), std::end(w), std::begin(w), square{});
}
\end{lstlisting}
\end{block}
\end{itemize}
\end{frame}

\begin{frame}[t,fragile]{Why function objects?}
\begin{itemize}
  \item Simple uses can be obatained with a regular function.
\end{itemize}
\begin{block}{Passing a function to an algorithm}
\begin{lstlisting}
double square(double x) noexcept { return x*x; }

void f(const std::vector<double> & v, std::vector<double> & w) {
  std::transform(std::begin(v), std::end(v), std::back_inserter(w), square);
  //...
}
\end{lstlisting}
\end{block}

\mode<presentation>{\vfill\pause}
\begin{itemize}
  \item Why using function objects?
    \begin{itemize}
      \item The may have input state.
      \item They may have output state.
      \item Better passing an object than passing a function pointer.
    \end{itemize}
\end{itemize}

\end{frame}

\begin{frame}[t,fragile]{Parametrized function objects}
\begin{itemize}
  \item A function object may store parameters that cannot be passed
        as arguments to a function
\end{itemize}
\begin{block}{Linear transformation}
\begin{lstlisting}
struct linear_transform {
  double a, b;
  double operator()(double x) const noexcept { return a * x + b; }
};

auto f(const std::vector<double> & v, std::vector<double> & w) {
  std::transform(std::begin(v), std::end(v), std::back_inserter(w), 
      linear_transform{2.5, 7.3});
}
\end{lstlisting}
\end{block}

\begin{itemize}
  \item \cppid{std::transform()} requires a function taking one argument.
\end{itemize}
\end{frame}

\begin{frame}[t,fragile]{Functions objects with state}
\begin{itemize}
  \item A function object may store and update state.
\end{itemize}
\begin{block}{Adding values in a vector}
\begin{lstlisting}
struct adder {
  double sum = 0.0;
  double operator()(double x) noexcept { sum += x; }
}

auto f(const std::vector<double> & v) {
  adder a{0};
  std::for_each(std::begin(v), std::end(v), a); 
  return a.sum;
}
\end{lstlisting}
\end{block}
\begin{itemize}
  \item \textbad{Note}: This is better done with \cppid{std::reduce()}.
\end{itemize}
\end{frame}


\begin{frame}[t,fragile]{Lambda expression}
\begin{itemize}
  \item A mechanism to specify a function object.
  \item Simple way of defining and passing a function to another context.
\end{itemize}

\begin{block}{Adding values in a vector}
\begin{lstlisting}
void f(const std::vector<double< & v, std::vector<double> & w) {
  std::transform(v.begin(), v.end(), std::back_inserter(w),
                 [](double x) { return x*x; });
}
\end{lstlisting}
\end{block}
\begin{itemize}
  \item The expression
\begin{lstlisting}
[](double x) { return x*x; }
\end{lstlisting}
  \item is a \textmark{lambda expression}.
\end{itemize}
\end{frame}

\begin{frame}[t,fragile]{Anatomy of lambda expression}
\begin{lstlisting}
std::transform(v.begin(), v.end(), std::back_inserter(w),
  [](double x) { 
    return x*x; 
  }
);
\end{lstlisting}

\mode<presentation>{\vfill}
\begin{itemize}
  \pause
  \item \textmark{Lambda introducer}: 
        Marks the beginning of a lambda expression.
\begin{lstlisting}[escapechar=@]
@{\color{red}[]}@(double x) {
\end{lstlisting}

  \pause
  \item \textmark{Lambda parameters}: 
        Equivalent to function parameters.
\begin{lstlisting}[escapechar=@]
[]@{\color{red}(double x)}@ {
\end{lstlisting}

  \pause
  \item \textmark{Lambda body}: 
        Equivalent to function body.
\begin{lstlisting}[escapechar=@]
[](double x) @{\color{red}\{ return x*x; \}}
\end{lstlisting}

  \pause
  \item \textmark{Return type}: 
        Usually automatically deduced.
\end{itemize}
\end{frame}

