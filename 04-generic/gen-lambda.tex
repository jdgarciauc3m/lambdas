\section{Generic lambdas (C++14)}

\begin{frame}[t]{Example: Printing a container in C++11}
\mode<presentation>{
\begin{columns}[T]
\column{.5\textwidth}
\lstinputlisting[lastline=14]{04-generic/gen-lambda/src/print_cont.cpp}
\column{.5\textwidth}
\lstinputlisting[firstline=16]{04-generic/gen-lambda/src/print_cont.cpp}
\end{columns}
}
\mode<article>{
\lstinputlisting{04-generic/gen-lambda/src/print_cont.cpp}
}
\end{frame}

\begin{frame}[t,fragile]{Polymorphic lambdas}
\begin{itemize}
  \item Allow writing a lambda expression where one or more parameters are \textmark{generic}.
    \begin{itemize}
      \item Use \cppkey{auto}.
    \end{itemize}
\end{itemize}
\pause
\begin{lstlisting}
std::sort(v.begin(), v.end(), [](const auto & x, const auto & y) {
  return x < y;
});
\end{lstlisting}

\mode<presentation>{\vfill\pause}
\begin{itemize}
  \item Generates a class with a \textmark{templated} \textgood{call operator}.
\end{itemize}
\begin{lstlisting}
class compiler_generated {
public:
  template <typename T1, typename T2>
  bool operator()(const T1 & x, const T2 & y) const {
    return x < y;
  }
\end{lstlisting}
\end{frame}


\begin{frame}[t]{Example: Printing a container in C++14}
\mode<presentation>{
\begin{columns}[T]
\column{.5\textwidth}
\lstinputlisting[lastline=14]{04-generic/gen-lambda/src/print_cont_gen.cpp}
\column{.5\textwidth}
\lstinputlisting[firstline=16]{04-generic/gen-lambda/src/print_cont_gen.cpp}
\end{columns}
}
\mode<article>{
\lstinputlisting{04-generic/gen-lambda/src/print_cont_gen.cpp}
}
\end{frame}
