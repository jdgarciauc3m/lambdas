\section{Exercise \arabic{section}}
\label{ej:lambdafiltro}

Given the following \cppid{customer} structure:
\begin{lstlisting}
struct customer {
  int id;
  std::string name;
  std::string family_name;

  customer(int i, std::string const &n, std::string const &fn)
      : id{i}, name{n}, family_name{fn} {}
};
\end{lstlisting}

Write a function \cppid{filter\_and\_sort} that takes
\begin{itemize}
\item A \cppid{vector} with customer information.
\item A generic predicate on a customer.
\item A generic comparison criteria between two customers.
\end{itemize}

The function should return a new \cppid{vector} where only customers that satisfy the predicate are stored. Such a vector should be sorted according to the comparison criteria.

Complete the call \cppid{filter\_and\_sort} in the main program so that
the program prints sorted by family name, and the by name all the people in the list
whose \cppid{id} is greater than \cppid{1}.
Use lambda expressions for passing the criteria to the function \cppid{filter\_and\_sort()}.
\begin{lstlisting}
int main() {
  std::vector<customer> v;
  v.emplace_back(3, "David", "Wheeler");
  v.emplace_back(11, "J. Daniel", "Garcia");
  v.emplace_back(2, "Bjarne", "Stroustrup");
  v.emplace_back(0, "Herb", "Sutter");
  v.emplace_back(5, "Carlos", "Garcia");

  auto r = filter_and_sort(v, /* Fill in the arguments*/);

  for (auto const &c : r) {
    std::cout << "Id: " << c.id;
    std::cout << " Name: " << c.name;
    std::cout << " Family name: " << c.family_name << "\n";
  }
}
\end{lstlisting}

\section{Exercise \arabic{section}}

Rewrite exercise~\ref{ej:lambdafiltro} using objects of
type \cppid{std::function} as parameters of function \cppid{filter\_and\_sort()}
and avoiding \cppkey{templates}.

\section{Exercise \arabic{section}}

Write a function taking two iterators and a set of operations.
The function shall traverse all elements in the sequence and apply
sequentially each of the operations.

Use the function to transform a vector of strings and for each string
remove vowels and convert the resulting string to uppercase as shown below:

\begin{lstlisting}
#include <vector>
#include <string>
#include <algorithm>
#include <iostream>

template <typename T, typename ... Ops>
void apply_comp_helper(T & v, Ops ... ops);

template <typename T, typename Op, typename ... Ops>
void apply_comp_helper(T & v, Op op, Ops ... ops) {
  op(v);
  if (sizeof...(ops) > 0)
    apply_comp_helper(v, ops...);
}

template <typename T>
void apply_comp_helper(T & v) {}

template <typename It, typename ... Ops>
void apply_comp(It first, It last, Ops ... ops) {
  for (auto i=first; i!=last; ++i) {
    apply_comp_helper(*i, ops...);
  }
}

std::string remove_vowels(const std::string & s) {
  std::string r;
  r.reserve(s.size());
  for (auto c : s) {
    switch (c) {
      case 'a': case 'A':
      case 'e': case 'E':
      case 'i': case 'I':
      case 'o': case 'O':
      case 'u': case 'U':
        break;
      default:
        r.push_back(c);
    }
  }
  return r;
}

int main() {
  std::vector<std::string> v;
  v.push_back("Daniel");
  v.push_back("Carlos");
  v.push_back("Jose");
  v.push_back("Manuel");
  apply_comp(v.begin(), v.end(),
             [](std::string & s) { s = remove_vowels(s); },
             [](std::string & s) {
               std::transform(s.begin(), s.end(), s.begin(), toupper);
             }
  );
  for (auto x : v) {
    std::cout << x << "\n";
  }
}
\end{lstlisting}
